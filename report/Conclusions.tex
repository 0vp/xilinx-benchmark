\section{Conclusions}

The emulation tests provided valuable insights into system behavior, but they do not fully capture real-world performance due to significant overhead and slower execution times. While emulation is useful for functional verification, it falls short in accurately reflecting actual hardware performance. The trends observed in hardware tests, such as consistent transfer speeds for large data sizes and similar rates for consecutive transfers, reinforce the reliability of the hardware measurements over emulation results. Therefore, while emulation offers a preliminary understanding, hardware testing is essential for precise performance evaluation. \\

Based on the benchmarking results, it is clear that both OpenCL (OCL) and Xilinx Runtime (XRT) APIs have their own strengths in data transfers. For small data sizes, XRT consistently outperforms OCL due to lower overhead, making it a more efficient choice for minimizing latency. However, for larger data sizes, both APIs achieve similar peak transfer speeds, indicating that OCL's overhead becomes negligible as data size increases. This suggests that while XRT offers an edge in scenarios requiring rapid, small transfers, OCL is equally viable for larger data transfers when peak throughput is the primary concern. \\

These findings have significant implications for FPGA-based application development. Choosing between OpenCL and XRT can greatly impact performance, especially in applications where latency and throughput are critical. For instance, in high-frequency trading, where microseconds matter, the lower overhead of XRT for small data transfers could provide a competitive edge. Similarly, if the data transfers are large and a fast development time is needed, then OCL might be the right choice as it is inherently easier and the overhead becomes negligible. This study aids in making informed decisions, ensuring that the chosen technology aligns with the specific requirements of a given application.